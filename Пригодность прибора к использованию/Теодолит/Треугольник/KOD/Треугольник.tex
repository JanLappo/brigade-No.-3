\documentclass[a4paper]{article}
\usepackage[14pt]{extsizes}
\usepackage{setspace,amsmath}
\usepackage{savesym}
\savesymbol{iint}
\usepackage{txfonts}
\restoresymbol{TXF}{iint}
\usepackage[left=20mm, top=20mm, right=20mm, bottom=20mm]{geometry}
\setlength{\parindent}{12,5mm}
\linespread{1.15}
\usepackage[russian]{babel}
\usepackage[T2A]{fontenc} % кодировка
\usepackage{fontspec} 
\usepackage{multirow}
\defaultfontfeatures{Ligatures={TeX}, Renderer=Basic} 
\setmainfont[Ligatures={TeX,Historic}]{Times New Roman}

\def\ang#1#2#3{$#1^\circ\,#2'\,#3''$} 

\usepackage{graphicx}
\usepackage{lscape}
\graphicspath{{pictures}}
\DeclareGraphicsExtensions{.pdf,.png,.jpg}
\begin{document}
\begin{center}
    \large{\textbf{Треугольник}}
\end{center}
\large{
\par Для проверки невзки в полигоне были выполнены измерения углов в треугольном полигоне. Для этого был построен равносторонний треугольник со сторонами 60 метров. Результаты измерений представленный в таблице ниже.\\
Таблица 1 -- Журнал измерения углов\\
\vspace{2mm}
\begin{tabular}{|c|c|c|c|c|}
\hline
Номер  & Положение  & \multirow{2}{*}{Отсчёт} & \multirow{2}{*}{Угол} & Средний \\
станции &  круга &  &  &  угол\\
\hline
\multirow{4}{*}{1} & \multirow{2}{*}{КЛ}  & \ang{0}{01}{26} & \multirow{2}{*}{\ang{66}{00}{06}} & \multirow{4}{*}{\ang{66}{00}{09}}\\
\cline{3-3}
 & & \ang{66}{01}{32} &  &  \\ 
\cline{2-4}
 & \multirow{2}{*}{КП} & \ang{180}{01}{32} & \multirow{2}{*}{\ang{66}{00}{11}} &  \\
\cline{3-3} 
 &  & \ang{246}{01}{43} &  &  \\
\hline

\multirow{4}{*}{2} & \multirow{2}{*}{КЛ}  & \ang{99}{43}{16} & \multirow{2}{*}{\ang{60}{13}{38}} & \multirow{4}{*}{\ang{60}{13}{43}}\\
\cline{3-3}
 & & \ang{159}{56}{54} &  &  \\
\cline{2-4}
 & \multirow{2}{*}{КП} & \ang{299}{43}{11} & \multirow{2}{*}{\ang{60}{13}{47}} &  \\
\cline{3-3}
 &  & \ang{359}{56}{58} &  &  \\
\hline

\multirow{4}{*}{3} & \multirow{2}{*}{КЛ}  & \ang{146}{40}{59} & \multirow{2}{*}{\ang{53}{46}{09}} & \multirow{4}{*}{\ang{53}{46}{03}}\\
\cline{3-3}
 & & \ang{200}{27}{08} &  &  \\
\cline{2-4}
 & \multirow{2}{*}{КП} & \ang{326}{41}{11} & \multirow{2}{*}{\ang{53}{45}{56}} &  \\
\cline{3-3}
 &  & \ang{20}{27}{07} &  &  \\
\hline
\end{tabular}
\vspace{2mm}
\par Невязка в треугольнике:
\begin{center}
$f_\beta = \sum\beta_n - \sum\beta_T$\\
$f_\beta = \ang{66}{00}{09} + \ang{60}{13}{43} + \ang{53}{46}{03} - 180^\circ = 5 ^{\prime\prime}$
\end{center}
\par Допустимая невязка в треугольнике:
\begin{center}
$f_{\beta{dop}} = 2\cdot t \cdot\sqrt{n}$\\
$f_{\beta{dop}} = 2\cdot 2^{\prime\prime} \cdot \sqrt{3} = 6,9^{\prime\prime} $
\end{center}
\par \textbf{Вывод:} фактическая невязка не превышает допустимую.
}

\end{document}
